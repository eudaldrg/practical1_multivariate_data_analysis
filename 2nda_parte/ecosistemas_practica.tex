\documentclass[]{article}
\usepackage{lmodern}
\usepackage{amssymb,amsmath}
\usepackage{ifxetex,ifluatex}
\usepackage{fixltx2e} % provides \textsubscript
\ifnum 0\ifxetex 1\fi\ifluatex 1\fi=0 % if pdftex
  \usepackage[T1]{fontenc}
  \usepackage[utf8]{inputenc}
\else % if luatex or xelatex
  \ifxetex
    \usepackage{mathspec}
  \else
    \usepackage{fontspec}
  \fi
  \defaultfontfeatures{Ligatures=TeX,Scale=MatchLowercase}
\fi
% use upquote if available, for straight quotes in verbatim environments
\IfFileExists{upquote.sty}{\usepackage{upquote}}{}
% use microtype if available
\IfFileExists{microtype.sty}{%
\usepackage{microtype}
\UseMicrotypeSet[protrusion]{basicmath} % disable protrusion for tt fonts
}{}
\usepackage[margin=1in]{geometry}
\usepackage{hyperref}
\hypersetup{unicode=true,
            pdftitle={Case of Study: Water Quality},
            pdfauthor={Eudald Romo y Laura Santulario Verdú},
            pdfborder={0 0 0},
            breaklinks=true}
\urlstyle{same}  % don't use monospace font for urls
\usepackage{color}
\usepackage{fancyvrb}
\newcommand{\VerbBar}{|}
\newcommand{\VERB}{\Verb[commandchars=\\\{\}]}
\DefineVerbatimEnvironment{Highlighting}{Verbatim}{commandchars=\\\{\}}
% Add ',fontsize=\small' for more characters per line
\usepackage{framed}
\definecolor{shadecolor}{RGB}{248,248,248}
\newenvironment{Shaded}{\begin{snugshade}}{\end{snugshade}}
\newcommand{\KeywordTok}[1]{\textcolor[rgb]{0.13,0.29,0.53}{\textbf{#1}}}
\newcommand{\DataTypeTok}[1]{\textcolor[rgb]{0.13,0.29,0.53}{#1}}
\newcommand{\DecValTok}[1]{\textcolor[rgb]{0.00,0.00,0.81}{#1}}
\newcommand{\BaseNTok}[1]{\textcolor[rgb]{0.00,0.00,0.81}{#1}}
\newcommand{\FloatTok}[1]{\textcolor[rgb]{0.00,0.00,0.81}{#1}}
\newcommand{\ConstantTok}[1]{\textcolor[rgb]{0.00,0.00,0.00}{#1}}
\newcommand{\CharTok}[1]{\textcolor[rgb]{0.31,0.60,0.02}{#1}}
\newcommand{\SpecialCharTok}[1]{\textcolor[rgb]{0.00,0.00,0.00}{#1}}
\newcommand{\StringTok}[1]{\textcolor[rgb]{0.31,0.60,0.02}{#1}}
\newcommand{\VerbatimStringTok}[1]{\textcolor[rgb]{0.31,0.60,0.02}{#1}}
\newcommand{\SpecialStringTok}[1]{\textcolor[rgb]{0.31,0.60,0.02}{#1}}
\newcommand{\ImportTok}[1]{#1}
\newcommand{\CommentTok}[1]{\textcolor[rgb]{0.56,0.35,0.01}{\textit{#1}}}
\newcommand{\DocumentationTok}[1]{\textcolor[rgb]{0.56,0.35,0.01}{\textbf{\textit{#1}}}}
\newcommand{\AnnotationTok}[1]{\textcolor[rgb]{0.56,0.35,0.01}{\textbf{\textit{#1}}}}
\newcommand{\CommentVarTok}[1]{\textcolor[rgb]{0.56,0.35,0.01}{\textbf{\textit{#1}}}}
\newcommand{\OtherTok}[1]{\textcolor[rgb]{0.56,0.35,0.01}{#1}}
\newcommand{\FunctionTok}[1]{\textcolor[rgb]{0.00,0.00,0.00}{#1}}
\newcommand{\VariableTok}[1]{\textcolor[rgb]{0.00,0.00,0.00}{#1}}
\newcommand{\ControlFlowTok}[1]{\textcolor[rgb]{0.13,0.29,0.53}{\textbf{#1}}}
\newcommand{\OperatorTok}[1]{\textcolor[rgb]{0.81,0.36,0.00}{\textbf{#1}}}
\newcommand{\BuiltInTok}[1]{#1}
\newcommand{\ExtensionTok}[1]{#1}
\newcommand{\PreprocessorTok}[1]{\textcolor[rgb]{0.56,0.35,0.01}{\textit{#1}}}
\newcommand{\AttributeTok}[1]{\textcolor[rgb]{0.77,0.63,0.00}{#1}}
\newcommand{\RegionMarkerTok}[1]{#1}
\newcommand{\InformationTok}[1]{\textcolor[rgb]{0.56,0.35,0.01}{\textbf{\textit{#1}}}}
\newcommand{\WarningTok}[1]{\textcolor[rgb]{0.56,0.35,0.01}{\textbf{\textit{#1}}}}
\newcommand{\AlertTok}[1]{\textcolor[rgb]{0.94,0.16,0.16}{#1}}
\newcommand{\ErrorTok}[1]{\textcolor[rgb]{0.64,0.00,0.00}{\textbf{#1}}}
\newcommand{\NormalTok}[1]{#1}
\usepackage{graphicx,grffile}
\makeatletter
\def\maxwidth{\ifdim\Gin@nat@width>\linewidth\linewidth\else\Gin@nat@width\fi}
\def\maxheight{\ifdim\Gin@nat@height>\textheight\textheight\else\Gin@nat@height\fi}
\makeatother
% Scale images if necessary, so that they will not overflow the page
% margins by default, and it is still possible to overwrite the defaults
% using explicit options in \includegraphics[width, height, ...]{}
\setkeys{Gin}{width=\maxwidth,height=\maxheight,keepaspectratio}
\IfFileExists{parskip.sty}{%
\usepackage{parskip}
}{% else
\setlength{\parindent}{0pt}
\setlength{\parskip}{6pt plus 2pt minus 1pt}
}
\setlength{\emergencystretch}{3em}  % prevent overfull lines
\providecommand{\tightlist}{%
  \setlength{\itemsep}{0pt}\setlength{\parskip}{0pt}}
\setcounter{secnumdepth}{0}
% Redefines (sub)paragraphs to behave more like sections
\ifx\paragraph\undefined\else
\let\oldparagraph\paragraph
\renewcommand{\paragraph}[1]{\oldparagraph{#1}\mbox{}}
\fi
\ifx\subparagraph\undefined\else
\let\oldsubparagraph\subparagraph
\renewcommand{\subparagraph}[1]{\oldsubparagraph{#1}\mbox{}}
\fi

%%% Use protect on footnotes to avoid problems with footnotes in titles
\let\rmarkdownfootnote\footnote%
\def\footnote{\protect\rmarkdownfootnote}

%%% Change title format to be more compact
\usepackage{titling}

% Create subtitle command for use in maketitle
\newcommand{\subtitle}[1]{
  \posttitle{
    \begin{center}\large#1\end{center}
    }
}

\setlength{\droptitle}{-2em}
  \title{Case of Study: Water Quality}
  \pretitle{\vspace{\droptitle}\centering\huge}
  \posttitle{\par}
  \author{Eudald Romo y Laura Santulario Verdú}
  \preauthor{\centering\large\emph}
  \postauthor{\par}
  \predate{\centering\large\emph}
  \postdate{\par}
  \date{05/21/2018}


\begin{document}
\maketitle

\subsection{Prior Analysis}\label{prior-analysis}

Let's first examine the main characteristics of our data, to base the
following points of our study on them.

As stated in the dataset information slides, the data represents the
contamination of sea ecosystems at different points. We have 57
different data points and each of them provides information for 7
different variables.

All variables are real non-negative numbers and some of them clearly
represent similar kinds of contamination. For instance,
\textbf{Colif\_total}, \textbf{Colif\_fecal}, and \textbf{Estrep\_fecal}
represent all organic contamination and are highly correlated, as seen
below.

Lastly, it is important to note that there's a really reduced number of
samples. The size of the data will be further analized when deciding
upon a discriminant for the discriminant analysis, but a first
observation is that no cluster analysis is going to be very robust or
meaningful if any group has less than 10 variables, this means that it
is highly unlikely that we have to analyse our groups hypothesis for
more than around 6 groups (we'll do it anyway for up until 8 groups for
the sake of illustrating what we qualitatively observed by using the
standard indicators for clustering quality.).

\begin{verbatim}
##              Colif_total Colif_fecal Estrep_fecal
## Colif_total        1.000       0.988        0.841
## Colif_fecal        0.988       1.000        0.861
## Estrep_fecal       0.841       0.861        1.000
\end{verbatim}

With this information, we can already choose a distance matrix.

\subsection{Distance matrix}\label{distance-matrix}

Our dataset clearly does not intend to evaluate profiles, it just
characterizes different samples of a certain set of variables of
interest to (presumably) check similatities or dissimilarities at
certain points. Furthermore, no specific value of any of the variables
has more weight than others, in particular, it does not make any
difference if there is a really low level of contamination of any kind
in a certain point or if there's none (handling the case where a
variable has 0 value only makes sense in situations where we are
studying variables like human behaviour, where theres a clear difference
between showing a slight interest in a topic or showing none).

This leaves out with the choice of using either Euclidean or Mahalanobis
distances. The high correlations shown above would advocate in favour of
the Mahalanobis one, but it was explained in one of the practical
sessions that in the original study that used this dataset, it was
intended that certain aspects of the ecosystem, like organic
contamination, had a higher weight than others. Because of that, we
decided to keep those weights and use the Euclidean distance. As the
variables have different units (and in particular, quite different
means/standard deviations) it is important to standardize the data
before computing the distances. Below is a subset of the obtained
distance matrix.

\begin{verbatim}
##          1        2        3        4        5
## 1 0.000000 5.183625 5.618755 6.067342 3.350898
## 2 5.183625 0.000000 2.471269 5.817088 3.721250
## 3 5.618755 2.471269 0.000000 3.787447 3.037862
## 4 6.067342 5.817088 3.787447 0.000000 3.665658
## 5 3.350898 3.721250 3.037862 3.665658 0.000000
\end{verbatim}

\subsection{Group Structure}\label{group-structure}

We're going to both do a first qualitative analysis of the group
structure through hierarchical clustering methods and then corroborate
and refine our first results through a more objective analysis using
clustering quality indicators as \(\Delta TESS\), \(pseudoF\), and
\(silhouette\) for 1 to 8 groups (as we already explained before, more
than 5 groups would result in some groups having less than 10 samples,
thus makin any further inference or discriminant analysis really poor.).

\textbf{Hierarchical Clustering}

At class we have considered the \emph{single}, \emph{complete},
\emph{average (UPGMA)} and \emph{ward.D2} methods. In this report, we
chose to use the \emph{ward.D2} as it tries to minimize the variance.
This is coherent with most analysis that use Euclidean distance, like
PCA. We expect this method to be more congruent with the distance chosen
than others like \emph{UPGMA}, which don't even need a fully defined
distance, just a dissimilarity matrix, which doesn't even need to
fulfill the triangular inequallity. We know that the hierarchichal
clustering can be subjective, as the election of the clustering method
can be biased or even have some arbitrariety. So, for the sake of
completeness, we are providing two different hierarchical analysis,
\emph{ward.D2} and \emph{UPGMA}.

As can be seen below, a much clearer analysis can be done in the
\emph{ward.D2} clustering, and it seems that the most possible choices
for number of groups are 2 or 3.

\includegraphics{ecosistemas_practica_files/figure-latex/unnamed-chunk-4-1.pdf}

\textbf{Non - Hierarchical Clustering}

For the non-herarchical clustering we will use \emph{kmeans} to obtain
intra and intercluster data. As said before, we will handle the cases
from 1 to 8 groups and we will compute the indicators mentioned before
for the situations in which it makes sense. sult among different values
of \emph{k}.

As can be seen below, the hierarchical analysis is ratified by the
non-hierarchical one. For all indicators, 2 and 3 groups have a higher
score (in particular, the difference is specially big for
\(\Delta TESS\)). Specifically, all the indicators for 2 groups are
higher than for 3. Furthermore, the plot of the individual silhouettes
for individual elements of each group clearly shows that the difference
between 2 and 3 groups is likely to be a splitting of the biggest group
into two smaller ones. This splitting reduces the average silhouette of
the smallest group and the silhouettes of the two new groups are also
smaller than the silhouette of the original one. Thus, all seems to
indicate that increasing the number of clusters to 3 results in a poorer
quality cluster structure. Because of that we will use 2 groups for the
remaining of this study.

\begin{Shaded}
\begin{Highlighting}[]
\KeywordTok{require}\NormalTok{(cclust)}
\end{Highlighting}
\end{Shaded}

\begin{verbatim}
## Loading required package: cclust
\end{verbatim}

\begin{Shaded}
\begin{Highlighting}[]
\NormalTok{results<-}\KeywordTok{data.frame}\NormalTok{()}
\ControlFlowTok{for}\NormalTok{(x }\ControlFlowTok{in} \KeywordTok{c}\NormalTok{ (}\DecValTok{1}\NormalTok{,}\DecValTok{2}\NormalTok{,}\DecValTok{3}\NormalTok{,}\DecValTok{4}\NormalTok{,}\DecValTok{5}\NormalTok{,}\DecValTok{6}\NormalTok{,}\DecValTok{7}\NormalTok{,}\DecValTok{8}\NormalTok{))\{}
\NormalTok{  result.km.x =}\StringTok{ }\KeywordTok{kmeans}\NormalTok{(countries2, }\DataTypeTok{centers=}\NormalTok{x, }\DataTypeTok{nstart=}\DecValTok{1000}\NormalTok{)}
\NormalTok{  Silh.km.x<-}\KeywordTok{silhouette}\NormalTok{(result.km.x}\OperatorTok{$}\NormalTok{cluster,}\KeywordTok{dist}\NormalTok{(countries2))}
\NormalTok{  results[(x),}\DecValTok{1}\NormalTok{]<-x}
\NormalTok{  results[(x),}\DecValTok{2}\NormalTok{]<-result.km.x}\OperatorTok{$}\NormalTok{totss}
\NormalTok{  results[(x),}\DecValTok{3}\NormalTok{]<-result.km.x}\OperatorTok{$}\NormalTok{tot.withinss}

  \ControlFlowTok{if}\NormalTok{ (x }\OperatorTok{<}\StringTok{ }\DecValTok{9} \OperatorTok{&&}\StringTok{ }\NormalTok{x }\OperatorTok{!=}\StringTok{ }\DecValTok{1}\NormalTok{) \{}
\NormalTok{    PseudoF.km.x<-}\KeywordTok{clustIndex}\NormalTok{(result.km.x, countries2, }\DataTypeTok{index=}\StringTok{"calinski"}\NormalTok{)}
\NormalTok{    results[(x),}\DecValTok{4}\NormalTok{]<-PseudoF.km.x}
\NormalTok{  \} }\ControlFlowTok{else}\NormalTok{ \{}
\NormalTok{    results[x,}\DecValTok{4}\NormalTok{] <-}\StringTok{ }\OtherTok{NA}
\NormalTok{  \}}
  
  \ControlFlowTok{if}\NormalTok{ (x }\OperatorTok{==}\StringTok{ }\DecValTok{1}\NormalTok{) \{}
\NormalTok{    results[(x),}\DecValTok{5}\NormalTok{]<-}\StringTok{ }\OtherTok{NA}
\NormalTok{    results[(x),}\DecValTok{6}\NormalTok{]<-}\StringTok{ }\OtherTok{NA}
\NormalTok{  \} }\ControlFlowTok{else}\NormalTok{ \{}
\NormalTok{    Overall.Silh.km.x<-}\KeywordTok{mean}\NormalTok{(Silh.km.x[,}\DecValTok{3}\NormalTok{])}
\NormalTok{    results[(x),}\DecValTok{5}\NormalTok{]<-Overall.Silh.km.x}
\NormalTok{    results[(x),}\DecValTok{6}\NormalTok{]<-}\StringTok{ }\NormalTok{(results[(x}\OperatorTok{-}\DecValTok{1}\NormalTok{),}\DecValTok{3}\NormalTok{] }\OperatorTok{-}\StringTok{ }\NormalTok{results[(x),}\DecValTok{3}\NormalTok{]) }\OperatorTok{/}\StringTok{ }\NormalTok{results[(x}\OperatorTok{-}\DecValTok{1}\NormalTok{),}\DecValTok{3}\NormalTok{]}
\NormalTok{  \}}
  \ControlFlowTok{if}\NormalTok{ (x }\OperatorTok{!=}\StringTok{ }\DecValTok{1} \OperatorTok{&&}\StringTok{ }\NormalTok{x }\OperatorTok{<}\StringTok{ }\DecValTok{4}\NormalTok{) }\KeywordTok{plot}\NormalTok{(Silh.km.x)}
\NormalTok{\}}
\end{Highlighting}
\end{Shaded}

\includegraphics{ecosistemas_practica_files/figure-latex/unnamed-chunk-5-1.pdf}
\includegraphics{ecosistemas_practica_files/figure-latex/unnamed-chunk-5-2.pdf}

\begin{Shaded}
\begin{Highlighting}[]
\KeywordTok{colnames}\NormalTok{(results)<-}\KeywordTok{c}\NormalTok{(}\StringTok{"K"}\NormalTok{,}\StringTok{"TOTSS"}\NormalTok{,}\StringTok{"WITHINSS"}\NormalTok{,}\StringTok{"PseudoF"}\NormalTok{,}\StringTok{"AvgSilh"}\NormalTok{,}\StringTok{"DeltaTESS"}\NormalTok{)}
\NormalTok{results}
\end{Highlighting}
\end{Shaded}

\begin{verbatim}
##   K TOTSS  WITHINSS  PseudoF   AvgSilh  DeltaTESS
## 1 1   392 392.00000       NA        NA         NA
## 2 2   392 235.39776 36.58966 0.3974478 0.39949551
## 3 3   392 168.38508 35.85592 0.2955183 0.28467849
## 4 4   392 142.55117 30.91472 0.2752161 0.15342161
## 5 5   392 122.64389 28.55119 0.2709973 0.13965007
## 6 6   392 109.48461 26.32020 0.2623816 0.10729669
## 7 7   392  96.46200 25.53147 0.2516608 0.11894469
## 8 8   392  87.14405 24.48809 0.2426378 0.09659709
\end{verbatim}

\section{\texorpdfstring{\texttt{\{r\}\ \#\ \#\ Compute\ K-means\ for\ a\ range\ of\ values\ of\ K\ \#\ require(vegan)\ \#\ result.cascadeKM\ =\ cascadeKM(countries2,\ inf.gr=2,\ sup.gr=10,\ iter\ =\ 1000,\ criterion\ ="calinski")\ \#\ attributes(result.cascadeKM)\ \#\ result.cascadeKM\$results\ \#\ result.cascadeKM\$size\ \#}}{\{r\} \# \# Compute K-means for a range of values of K \# require(vegan) \# result.cascadeKM = cascadeKM(countries2, inf.gr=2, sup.gr=10, iter = 1000, criterion ="calinski") \# attributes(result.cascadeKM) \# result.cascadeKM\$results \# result.cascadeKM\$size \#}}\label{r-compute-k-means-for-a-range-of-values-of-k-requirevegan-result.cascadekm-cascadekmcountries2-inf.gr2-sup.gr10-iter-1000-criterion-calinski-attributesresult.cascadekm-result.cascadekmresults-result.cascadekmsize}

\subsection{Groups characterization}\label{groups-characterization}

\textbf{Representatives characterization}

Below are shown the values of the representatives of each of the two
groups. The analysis has been done with standardized values, so the
representatives are shown accordingly. It can be seen that the first
group representative has relatively low values in all the contamination
indicators. The group 2 representative has overall positive values
(which mean higher than the mean values in the original data), in
particular for the organic contamination variables mentioned at the
beginning and in \textbf{DQO\_M} (which also determines organic
contamination). Hence, group 1 represents low contamination and group 2
represents high organic contamination. This will be expanded further
when analizing the groups in a 2 dimentional space, but we don't feel
confident enough yet to infer anything about the inorganic
contamination.

\begin{Shaded}
\begin{Highlighting}[]
\NormalTok{result.km.}\DecValTok{2}\NormalTok{ <-}\StringTok{ }\KeywordTok{kmeans}\NormalTok{(countries2, }\DataTypeTok{centers=}\DecValTok{2}\NormalTok{, }\DataTypeTok{nstart=}\DecValTok{1000}\NormalTok{)}
\NormalTok{result.km.}\DecValTok{2}\OperatorTok{$}\NormalTok{centers}
\end{Highlighting}
\end{Shaded}

\begin{verbatim}
##   Colif_total Colif_fecal Estrep_fecal Cont_mineral Conductivitat
## 1  -1.4652823  -1.4829914   -1.5853018   -0.4077678   -0.30475090
## 2   0.3907419   0.3954644    0.4227471    0.1087381    0.08126691
##   Solids_susp      DQO_M
## 1   -1.118891 -1.3900417
## 2    0.298371  0.3706778
\end{verbatim}

\begin{Shaded}
\begin{Highlighting}[]
\NormalTok{ecosystems}\OperatorTok{$}\NormalTok{cluster.km.}\DecValTok{2}\NormalTok{ <-}\StringTok{ }\NormalTok{result.km.}\DecValTok{2}\OperatorTok{$}\NormalTok{cluster}
\end{Highlighting}
\end{Shaded}

\textbf{Representation in a low dimensional space} We will use the
\emph{clusplot} command (which internally uses PCA) to draw the data, as
it provides nice plots. We will also compute the regular PCA of the data
so that we can interpret the principal components.

\includegraphics{ecosistemas_practica_files/figure-latex/unnamed-chunk-7-1.pdf}

Above are the PCA plots for 2 and three groups, together with a cluster
encircling purely for representation purposes. With only 2 dimensions
the explained variability is higher than 80\%, so we deem 2 dimentions
as sufficient.

For interpretation of the principal components, we look at the loadings
of the PCA. It seems that the first component (which accounts for around
60\% of the explained variability) strongly depends on most of the
organic contamination variables. Some inorganic contamination variables
(like mineral contamination and solids in suspension) also affect it but
to a lesser degree, and conductivity doesn't affect it in any
appreciable way. For this reason we only deem to conclude that PC1 grows
as organic contamination increases. On the other hand the PC2 seems to
be more related to the inorganic contamination as it grows importantly
when the mineral contamination decreases or when the conductivity
increases. As can be seen in the plot above, both groups are centered in
the y axis, so we don't feel confident enough to infer anything about
the inorganic contamination of each of the two groups.

\begin{verbatim}
## 
## Loadings:
##               Comp.1 Comp.2 Comp.3 Comp.4 Comp.5 Comp.6 Comp.7
## Colif_total    0.451  0.251                0.556         0.642
## Colif_fecal    0.459  0.239                0.393        -0.752
## Estrep_fecal   0.429  0.137  0.531 -0.306 -0.563  0.323       
## Cont_mineral   0.206 -0.547  0.423  0.691                     
## Conductivitat         0.656 -0.196  0.643 -0.338              
## Solids_susp    0.384 -0.320 -0.650        -0.107  0.559       
## DQO_M          0.459 -0.170 -0.261 -0.110 -0.304 -0.755  0.133
## 
##                Comp.1 Comp.2 Comp.3 Comp.4 Comp.5 Comp.6 Comp.7
## SS loadings     1.000  1.000  1.000  1.000  1.000  1.000  1.000
## Proportion Var  0.143  0.143  0.143  0.143  0.143  0.143  0.143
## Cumulative Var  0.143  0.286  0.429  0.571  0.714  0.857  1.000
\end{verbatim}

\begin{verbatim}
##    Comp.1 
## 0.5902417
\end{verbatim}

It is tempting to add a new group, given that it adds some inorganic
separation in the contaminated group, as can be seen in the figure
below. We think it is important to keep our analysis impartial and
follow the indicators that we computed before viewing the 2D
representation of the data, otherwise we could add bias to our study.
Also, the two contaminated groups in the 3-cluster plot seem to be hard
to distinguish, so this seems to agree with our Silhouettes analysis:
adding a new group will only decrease the quality of our group
structure. Thus we reject the 3-clusters hypothesis.

\begin{Shaded}
\begin{Highlighting}[]
\NormalTok{result.km.}\DecValTok{3}\NormalTok{ <-}\StringTok{ }\KeywordTok{kmeans}\NormalTok{(countries2, }\DataTypeTok{centers=}\DecValTok{3}\NormalTok{, }\DataTypeTok{nstart=}\DecValTok{1000}\NormalTok{)}
\KeywordTok{clusplot}\NormalTok{(countries2, result.km.}\DecValTok{3}\OperatorTok{$}\NormalTok{cluster, }\DataTypeTok{main =} \StringTok{"Kmeans plot, k = 3"}\NormalTok{, }\DataTypeTok{color =} \OtherTok{TRUE}\NormalTok{, }\DataTypeTok{labels=}\DecValTok{2}\NormalTok{)}
\end{Highlighting}
\end{Shaded}

\includegraphics{ecosistemas_practica_files/figure-latex/unnamed-chunk-9-1.pdf}

\subsection{Group distinction}\label{group-distinction}

We'll use an inference analysis to check whether the two obtained groups
can be really considered different groups.

We first check whether the groups follow normal multivariates using the
Anderson-Darling test. As can be seen below, we can assume they follow
it with a p-value of 0.05.

\begin{Shaded}
\begin{Highlighting}[]
\KeywordTok{require}\NormalTok{(mvnTest) }\CommentTok{# Multivariate normality test}
\end{Highlighting}
\end{Shaded}

\begin{verbatim}
## Loading required package: mvnTest
\end{verbatim}

\begin{verbatim}
## Loading required package: mvtnorm
\end{verbatim}

\begin{Shaded}
\begin{Highlighting}[]
\KeywordTok{require}\NormalTok{(biotools) }\CommentTok{# Covariance Homogeneity test}
\end{Highlighting}
\end{Shaded}

\begin{verbatim}
## Loading required package: biotools
\end{verbatim}

\begin{verbatim}
## Loading required package: rpanel
\end{verbatim}

\begin{verbatim}
## Loading required package: tcltk
\end{verbatim}

\begin{verbatim}
## Package `rpanel', version 1.1-4: type help(rpanel) for summary information
\end{verbatim}

\begin{verbatim}
## Loading required package: tkrplot
\end{verbatim}

\begin{verbatim}
## Loading required package: MASS
\end{verbatim}

\begin{verbatim}
## Loading required package: lattice
\end{verbatim}

\begin{verbatim}
## Loading required package: SpatialEpi
\end{verbatim}

\begin{verbatim}
## Loading required package: sp
\end{verbatim}

\begin{verbatim}
## ---
## biotools version 3.1
\end{verbatim}

\begin{verbatim}
## 
\end{verbatim}

\begin{Shaded}
\begin{Highlighting}[]
\KeywordTok{require}\NormalTok{(ICSNP) }\CommentTok{# T2}
\end{Highlighting}
\end{Shaded}

\begin{verbatim}
## Loading required package: ICSNP
\end{verbatim}

\begin{verbatim}
## Loading required package: ICS
\end{verbatim}

\begin{Shaded}
\begin{Highlighting}[]
\KeywordTok{require}\NormalTok{(Hotelling) }\CommentTok{# T2 Permutation Test}
\end{Highlighting}
\end{Shaded}

\begin{verbatim}
## Loading required package: Hotelling
\end{verbatim}

\begin{verbatim}
## Loading required package: corpcor
\end{verbatim}

\begin{Shaded}
\begin{Highlighting}[]
\NormalTok{g1<-}\KeywordTok{as.matrix}\NormalTok{(ecosystems[}\KeywordTok{which}\NormalTok{(ecosystems}\OperatorTok{$}\NormalTok{cluster}\OperatorTok{==}\DecValTok{1}\NormalTok{),}\OperatorTok{-}\DecValTok{8}\NormalTok{]) }\CommentTok{# Multivariate normality}
\KeywordTok{AD.test}\NormalTok{(g1, }\DataTypeTok{qqplot =} \OtherTok{FALSE}\NormalTok{)}
\end{Highlighting}
\end{Shaded}

\begin{verbatim}
##             Anderson-Darling test for Multivariate Normality 
## 
##   data : g1 
## 
##   AD              : 0.7579276 
##   p-value         : 0.819718 
## 
##   Result  : Data are multivariate normal (sig.level = 0.05)
\end{verbatim}

\begin{Shaded}
\begin{Highlighting}[]
\NormalTok{g2<-}\KeywordTok{as.matrix}\NormalTok{(ecosystems[}\KeywordTok{which}\NormalTok{(ecosystems}\OperatorTok{$}\NormalTok{cluster}\OperatorTok{==}\DecValTok{2}\NormalTok{),}\OperatorTok{-}\DecValTok{8}\NormalTok{]) }\CommentTok{# Multivariate normality}
\KeywordTok{AD.test}\NormalTok{(g2, }\DataTypeTok{qqplot =} \OtherTok{FALSE}\NormalTok{)}
\end{Highlighting}
\end{Shaded}

\begin{verbatim}
##             Anderson-Darling test for Multivariate Normality 
## 
##   data : g2 
## 
##   AD              : 0.4142843 
##   p-value         : 0.6517348 
## 
##   Result  : Data are multivariate normal (sig.level = 0.05)
\end{verbatim}

As they follow multivariate normals, we can test for the homogeneity of
their Variance/Covariance matrices and their mean vectors. As seen in
the chi-squared test below, the p-value for the homogeneity of the
variance-covariance matrix is really high, so we cannot reject the null
hypothesis of the variance-covariance matrix being homogenious.

\begin{Shaded}
\begin{Highlighting}[]
\KeywordTok{boxM}\NormalTok{(ecosystems[,}\OperatorTok{-}\DecValTok{8}\NormalTok{], ecosystems[,}\DecValTok{8}\NormalTok{])}
\end{Highlighting}
\end{Shaded}

\begin{verbatim}
## 
##  Box's M-test for Homogeneity of Covariance Matrices
## 
## data:  ecosystems[, -8]
## Chi-Sq (approx.) = 22.83, df = 28, p-value = 0.7415
\end{verbatim}

On the other hand, with a negligible p-value, we can reject the null
hypothesis of both groups having the same mean, as can be seen below.
With this, we can conclude that both groups are really different and our
analysis up to this point holds.

\begin{Shaded}
\begin{Highlighting}[]
\KeywordTok{HotellingsT2}\NormalTok{(g1, g2, }\DataTypeTok{test =} \StringTok{"chi"}\NormalTok{)}
\end{Highlighting}
\end{Shaded}

\begin{verbatim}
## 
##  Hotelling's two sample T2-test
## 
## data:  g1 and g2
## T.2 = 175.37, df = 7, p-value < 2.2e-16
## alternative hypothesis: true location difference is not equal to c(0,0,0,0,0,0,0)
\end{verbatim}

\subsection{Prediction}\label{prediction}

Lastly, we need want to predict the cluster that corresponds to a new
sample. In order to do that we decide to use either a linear or a
quadratic discriminant. The previous inference advocates in favour of
the linear one, as the covariance matrices seem to be the same (or, at
least, we cannot reject the possibility that they are the same).

Furthermore, we have \(57 * 7 = 399\) distinct pieces of data (seven for
each of the samples). If we were to use a linear discriminant, we would
need to determine \(2 * 7 + 7 * 8 / 2 = 42\) estimators (7 for each
element of the mean vector of a multinormal distribution of 7 variables
and \(7 * 8 / 2\) for the variance/covariance matrix). This leaves us
with a proportion of around 10 pieces of data per estimator, a low but
assumible amount. If we were to use a quadratic one, we would need to
determine \(2 * 7 + 2 * 7 * 8 / 2 = 70\) estimators, having only 5.7
pieces of data per estimator, clearly insufficient. Thus, we decide to
use a linear one (we also double checked the quadratic one out of
curiosity and it gave a lower prediction rate when using leave-one-out
crossvalidation).

Below we show the quality estimation of our linear discriminant.
According to this analysis, the discriminant should have a low error
rate of under 5\%.

\begin{Shaded}
\begin{Highlighting}[]
\NormalTok{fit.l.cv <-}\StringTok{ }\KeywordTok{lda}\NormalTok{(cluster.km.}\DecValTok{2} \OperatorTok{~}\StringTok{ }\NormalTok{., }\DataTypeTok{data =}\NormalTok{ ecosystems, }\DataTypeTok{na.action =} \StringTok{"na.omit"}\NormalTok{, }\DataTypeTok{CV =}\NormalTok{ T) }\CommentTok{# Linear and crossvalidated}
\KeywordTok{sum}\NormalTok{(}\KeywordTok{diag}\NormalTok{(}\KeywordTok{prop.table}\NormalTok{(}\KeywordTok{table}\NormalTok{(ecosystems}\OperatorTok{$}\NormalTok{cluster.km.}\DecValTok{2}\NormalTok{, fit.l.cv}\OperatorTok{$}\NormalTok{class))))}
\end{Highlighting}
\end{Shaded}

\begin{verbatim}
## [1] 0.9649123
\end{verbatim}

Finally, we predict the cluster that corresponds to the new point
(251,241,109,42,25,972,715). As can be seen below, the new point belongs
to the clusten number 2 (which corresponds to organically contaminated
samples) with almost total certainty. Looking at the values of the new
point, this result is congruent with our analysis (as this point has
higher-than-the-mean levels for most of the organic contamination
variables).

\begin{Shaded}
\begin{Highlighting}[]
\KeywordTok{require}\NormalTok{(MASS)}
\KeywordTok{require}\NormalTok{(ggplot2)}
\end{Highlighting}
\end{Shaded}

\begin{verbatim}
## Loading required package: ggplot2
\end{verbatim}

\begin{Shaded}
\begin{Highlighting}[]
\KeywordTok{require}\NormalTok{(vegan)}
\end{Highlighting}
\end{Shaded}

\begin{verbatim}
## Loading required package: vegan
\end{verbatim}

\begin{verbatim}
## Loading required package: permute
\end{verbatim}

\begin{verbatim}
## This is vegan 2.5-2
\end{verbatim}

\begin{Shaded}
\begin{Highlighting}[]
\KeywordTok{require}\NormalTok{(cluster)}
\KeywordTok{require}\NormalTok{(factoextra)}
\end{Highlighting}
\end{Shaded}

\begin{verbatim}
## Loading required package: factoextra
\end{verbatim}

\begin{verbatim}
## Welcome! Related Books: `Practical Guide To Cluster Analysis in R` at https://goo.gl/13EFCZ
\end{verbatim}

\begin{Shaded}
\begin{Highlighting}[]
\KeywordTok{require}\NormalTok{(clusterSim) }\CommentTok{# predict}
\end{Highlighting}
\end{Shaded}

\begin{verbatim}
## Loading required package: clusterSim
\end{verbatim}

\begin{verbatim}
## 
## This is package 'modeest' written by P. PONCET.
## For a complete list of functions, use 'library(help = "modeest")' or 'help.start()'.
\end{verbatim}

\begin{Shaded}
\begin{Highlighting}[]
\NormalTok{element <-}\StringTok{ }\KeywordTok{data.frame}\NormalTok{(}\DataTypeTok{Colif_total =} \DecValTok{251}\NormalTok{, }\DataTypeTok{Colif_fecal =} \DecValTok{241}\NormalTok{, }\DataTypeTok{Estrep_fecal =} \DecValTok{109}\NormalTok{, }\DataTypeTok{Cont_mineral =} \DecValTok{42}\NormalTok{, }\DataTypeTok{Conductivitat =} \DecValTok{25}\NormalTok{, }\DataTypeTok{Solids_susp =} \DecValTok{972}\NormalTok{, }\DataTypeTok{DQO_M =} \DecValTok{715}\NormalTok{)}
\CommentTok{# element <- element - apply(ecosystems[,-8], MARGIN = 2, mean)}
\CommentTok{# element <- element / apply(ecosystems[,-8], MARGIN = 2, sd)}

\CommentTok{# ecosystems$cluster <- as.factor(result.km.2$cluster)}
\NormalTok{fit.l.ncv <-}\StringTok{ }\KeywordTok{lda}\NormalTok{(cluster.km.}\DecValTok{2} \OperatorTok{~}\StringTok{ }\NormalTok{., }\DataTypeTok{data =}\NormalTok{ ecosystems, }\DataTypeTok{na.action =} \StringTok{"na.omit"}\NormalTok{, }\DataTypeTok{CV =}\NormalTok{ F) }\CommentTok{# Should use the non cv for prediction and the cv for analysis of the discriminant itself}
\CommentTok{# ct <- table(ecosystems$cluster, predict(fit.l.ncv,ecosystems[, -8])$class) }
\CommentTok{# sum(diag(prop.table(ct)))}
\KeywordTok{predict}\NormalTok{(fit.l.ncv, element)}
\end{Highlighting}
\end{Shaded}

\begin{verbatim}
## $class
## [1] 2
## Levels: 1 2
## 
## $posterior
##              1         2
## 1 6.268562e-06 0.9999937
## 
## $x
##        LD1
## 1 1.231824
\end{verbatim}

\begin{Shaded}
\begin{Highlighting}[]
\CommentTok{# fit.q.ncv <- qda(cluster ~ ., data = ecosystems, na.action = "na.omit", CV = F)}
\CommentTok{# sum(diag(prop.table(table(ecosystems$cluster, predict(fit.q.ncv, ecosystems[,-8])$class))))}
\CommentTok{# predict(fit.q.ncv, element)}

\CommentTok{# fit.q.cv <- qda(cluster ~ ., data = ecosystems, na.action = "na.omit", CV = T)}
\CommentTok{# sum(diag(prop.table(table(ecosystems$cluster, fit.q.cv$class))))}
\end{Highlighting}
\end{Shaded}


\end{document}
